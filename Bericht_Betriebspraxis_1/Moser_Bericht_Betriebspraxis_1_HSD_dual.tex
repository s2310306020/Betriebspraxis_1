%%% Template für den Bericht zur Betriebspraxis 
\documentclass[praktikum,german]{hgbthesis}
\usepackage[backend=biber,style=numeric,]{biblatex}
\usepackage{appendix}

\graphicspath{{images/}}    % wo liegen die Bilder?
\logofile{logo}	            % name of PDF, remove or use \logofile{} for no logo
\bibliography{references} 	% Angabe der BibTeX-Datei (bei Bedarf) (references.bib)

\DefineBibliographyStrings{german}{
  bibliography = {Quellen},
}

%%%----------------------------------------------------------
\begin{document}
%%%----------------------------------------------------------


\berichtNr{\uproman{1}}
\author{Jakob Moser}
\studiengang{HSD Dual}
\studienort{Hagenberg}
\abgabedatum{2025}{04}{01}
\betreuer{Ing. Stefan Knauseder \\ Leitung Entwicklung Hardware}
\firma{
	Ginzinger electronic systems GmbH
	Gewerbegebiet Pirath 16\\
	4952 Weng im Innkreis
}
\firmenTel{+43 7723 54 22}
\firmenUrl{www.ginzinger.com}


%%%----------------------------------------------------------
\frontmatter
\maketitle
\tableofcontents
%%%----------------------------------------------------------

\chapter{Kurzfassung}
% ===================

% \color{blue}   % Nur die Hinweise sind in blauer Schrift. Bitte alles in Schwarz halten.
% Umfang der Kurzfassung: ca.\ 200 Worte.
% \\
% \\
% Inhalt der Kurzfassung:
% \begin{itemize}
% \item In welchem Unternehmen wurde die Betriebspraxis absolviert (Standort, Branche, Tätigkeitsbereich)?
% \item In welcher Fachabteilung war man beschäftigt und was war die eigene Aufgabenstellung?
% \item Was war das Ziel und wie ist man vorgegangen?
% \item Was war das Ergebnis?
% \end{itemize}

% \vskip 15mm

% \noindent Zum allgemeinen Inhalt des Berichts und für die Beurteilung relevant:\\

% \begin{enumerate}
% 	\item Der technische Bericht gibt einen Einblick in die Betriebspraxis. Neben der Darstellung des Abteilung werden der eigene Aufgabenbereich und die Aufgabenstellung beschrieben. Im Anschluss wird die technisch-methodische Umsetzung erläutert und auf die Ergebnisse eingegangen.

% \item Ziel des Berichtes ist eine theoretische Reflexion der Betriebspraxis in bewusst kompakter Form, wo die essentiellen Tätigkeiten dargestellt werden. Der Bericht zeigt nicht nur \emph{was} und \emph{wie}, sondern auch \emph{warum} etwas realisiert wurde (Nutzen). 

% \item In technischen Berichten sind Formuliereungen in der Ich-Form zu vermeiden!

% \item Vor der Abgabe ist der Bericht unbedingt einer Rechtschreib- und Grammatikprüfung zu unterziehen!
% \end{enumerate}

% \color{black}

Die erste Betriebspraxis im Rahmen des dualen Studiums wurde bei der Firma Ginzinger electronic systems GmbH mit Hauptsitz in Weng im Innkreis absolviert.
Das Unternehmen ist Komplettanbieter für maßgefertigte Embedded Lösungen mit eigener Elektronikproduktion am Hauptsitz in Weng im Innkreis und Hardware- sowie Softwareentwicklung in Linz und Altheim. Als Komplettanbieter gehören zum Tätigkeitsbereich die Entwicklung von Embedded Systems für Energietechnik, Medizintechnik, Transportindustrie und anderen Branchen, EMS (Electronic Manufacturing Service) Dienstleistungen und Life Cycle Management also die Unterstützung von der ersten Idee, über den gesamten Produktlebenszyklus und darüber hinaus.\cite{Ginzinger}

Der Hauptbestandteil der Betriebspraxis fand in der Hardwareentwicklung statt, mit vereinzelten Aufgaben aus der Softwareabteilung. Die eigene Aufgabenstellung umfasste verschiedene Teilaufgaben aus laufenden Kundenprojekten. Dazu gehörten unter anderem die Inbetriebnahme und Verifikation eines HMI auf Basis eines i.MX8-Prozessors, die mechanische Konstruktion einer Produktionsvorrichtung, das Design einer Interposer-Platine zur Korrektur eines Layoutfehlers sowie EMV-Störfestigkeitsmessungen und Touchscreen-Konfigurationsoptimierungen.

Nach erfolgreicher Inbetriebnahme einer neuen Charge von HMI-Systemen zur Heizungssteuerung konnten umfangreiche Funktionstests und Verifizierungsmessungen durchgeführt werden, die Erkenntnisse über mögliche Problemstellen und Optimierungspotenziale lieferten. Darüber hinaus wurde eine kosteneffiziente Lösung zur Behebung eines Layoutfehlers entwickelt und es konnten durch gezielte Maßnahmen in mehreren Projekten signifikante Verbesserungen der EMV-Störfestigkeit erzielt werden.

Um alle Testergebnisse, Messungen und Arbeitsschritte nachvollziehbar und rekonstruierbar festzuhalten, wurden alle Arbeitsschritte und Ergebnisse im Firmeninternen GitLab in Form von Worklogs festgehalten.

%%%----------------------------------------------------------
\mainmatter           %Hauptteil (ab hier arab. Seitenzahlen)
%%%----------------------------------------------------------

\chapter{Die Fachabteilung}
%========================


% \color{blue}
% Hier wird die Fachabteilung vorgestellt, in der die Betriebspraxis absolviert wurde:
% \begin{itemize}
% 	\item Organisation der Fachabteilung
% 	\item Informationen zur Fachabteilung
% 	\item Informationen zum Tätigkeitsbereich der Fachabteilung
% \end{itemize}

% \vskip 8mm
% Umfang: ca. 1 Seite
% \color{black}

Die Betriebspraxis wurde in der Entwicklung am Standort Altheim mit Schwerpunkt auf Hardwareentwicklung und einzelnen Teilaufgaben aus der Softwareentwicklung absolviert. Die Entwicklung bei Ginzinger electronic systems ist unterteilt in Hardware- und Softwareentwicklung sowie der Prüfmittelentwicklung verteilt auf die Standorte Weng, Altheim und Linz. 
Geleitet wird die Entwicklung von Entwicklungsleiter DI Stefan Schöfegger welcher auch im Speziellen für die Leitung der Softwareentwicklung zuständig ist. Für die Hardwareentwicklung ist Ing. Stefan Kanuseder als Leiter Entwicklung Hardware zuständig. Trotz der Aufteilung auf unterschiedliche Standorte wird speziell bei den Entwicklungsstandorten großer Wert auf enge Zusammenarbeit zwischen den Entwicklungsabteilungen gelegt, dies ist vor allem in Kundenprojekten notwendig da diese meist durch jeweils einen Mitarbeiter aus Hard- und Softwareentwicklung umgesetzt werden. Durch die Organisation in Abteilungsübergreifende Gemeinschaftsbüros werden kurze Wege ermöglicht und ein aktiver Austausch zwischen den Mitarbeitenden gefördert. Standortübergreifend findet die Kommunikation großteils über Microsoft Teams statt, wo beispielsweise wöchentlich ein jour fixe der Hardwareentwicklung abgehalten wird um unter anderem den Status von Projekten, Neuheiten in der Abteilung oder auch neue Erkenntnisse in unterschiedlichen Bereichen besprochen werden können. 

Zur Ausstattung der Hardwareentwicklung am Standort Altheim gehört unter anderem eine EMV Nische für interne Messungen, ein Lötplatz für Nacharbeit, Modifikation und Reparatur von Baugruppen, mehrere Oszilloskope in unterschiedlichen Ausführungen, zwei Klimaschränke für Temperaturtests und noch wertere Ausstattung welche für die täglichen Aufgaben in der Hardwareentwicklung notwendig sind.

Der Tätigkeitsbereich der Hardwareentwicklung umfasst die Entwicklung von Hardwarelösungen im Embedded Linux Umfeld von der Konzepterstellung mit dem Kunden, über die Entwicklung der Designs bis hin zur Verifikation und Freigabe für die Produktion. In der Regel wird bei Kundenprojekten die Projektleitung durch einen Hardwareentwickler übernommen und das Projekt in Zusammenarbeit mit einem Softwareentwickler umgesetzt, durch diese Organisation in kleinen Projektteams gestalten sich die Aufgaben sehr abwechslungsreich was speziell für das Berufspraktikum eine Interessante und vielfältige Aufgabenpalette bietet.

\chapter{Ausgangssituation und Aufgabenstellung}
%===============================================
%\color{blue}
%Hier wird die Ausgangssituation in der Fachabteilung zu Beginn erläutert. Darauf aufbauend wird die eigene Aufgabenstellung geschildert. Ziel, Zweck und Nutzen stehen dabei im Mittelpunkt. Man erfährt hier nicht nur, \emph{was} die Aufgabe war, sondern auch \emph{welcher Zweck} damit verfolgt wurde.
%
%\vskip 8mm
%Umfang: ca. 2 Seiten
%\color{black}

Da bereits vor dem dualen Studium ein Dienstverhältnis bestand, waren viele Abläufe und Gegebenheiten im Unternehmen bereits vor der ersten Praxisphase des dualen Studiums bekannt. Aufgrund der begrenzten Zeit des Betriebspraktikums und der relativ kurzen Zugehörigkeit im Unternehmen wurden keine eigenen Kundenprojekte übernommen, sondern Großteiles Teilaufgaben aus unterschiedlichen Kundenprojekten übernommen. 

\section{Inbetriebnahme und Verifikation HMI}
Zu Beginn der Betriebspraxis wurden verschiedene Aufgaben für die Inbetriebnahme und Verifikation eines Bedienteils zur Heizungssteuerung auf Basis des i.MX8 umgesetzt.
Vor dem Start des ersten Praxisblocks wurde ein Skript zur automatisierten Inbetriebnahme und Funktionstestung der Baugruppen erstellt. Zu Beginn des Betriebspraktikums konnte eine neu eingetroffene Charge an Entwicklungsmustern mithilfe dieses Skript automatisiert in Betrieb genommen und getestet werden. Darüber hinaus umfassten die Aufgabenstellung Temperaturmessungen, die Überprüfung einiger Spannungspegel an unterschiedlichen Pins, die Erstellung eines Blockschaltbildes mit allen Funktionsblöcken sowie verschiedene Verifikationsmessungen, wie EMV-Störeinkopplung und die Messung des SDIO-Interfaces der SD-Karte.

\section{Mechanische Konstruktion einer Produktionsvorrichtung}
Im Rahmen eines Kundenprojekts war es erforderlich, LED-Module in unterschiedlichen Ausführungen zu produzieren. Diese Module sollten grundsätzlich aus demselben Aufbau bestehen. Als Basis dient eine relativ kleine, runde Leiterplatte, welche mit SMD-LEDs bestückt wird. Für die Kontaktierung soll die Leiterplatte mit einem Sockel verklebt werden, welcher über zwei stiftförmige Pins für die Versorgung der LEDs verfügt. Da für das Verkleben ein Kleber eingesetzt wird, welcher bei hohen Temperaturen aushärtet, müssen die Module zweimal den Reflow-Ofen durchlaufen. Zunächst wird die Leiterplatte mit den Sockeln verklebt, gefolgt von einem weiteren Durchlauf, um die SMD-LEDs und die Pins mit der Leiterplatte zu verlöten. Aufgrund der speziellen Form der Module und des besonderen Produktionsablaufs war es notwendig, eine gefräste Vorrichtung zu entwerfen. Diese Vorrichtung sollte es ermöglichen, die Sockel in der richtigen Ausrichtung im Reflow-Ofen zu verkleben und eine weite Vorrichtung um die Leiterplatten mit bereits verklebten Sockeln mit SMD-LEDs bestücken zu können.

\section{Design einer Interpser Platine}
Die Ausgangssituation dieser Aufgabenstellung war, dass bei einer Leiterplatte eines Kunden, welche bei Ginzinger bestückt werden sollte, ein Designfehler vorlag. Beim Design der Leiterplatte wurden RX- und TX-Leitungen an einem Stecker vertauscht. Da es sich um eine relativ große Leiterplatte mit 16 Lagen und mehreren impedanzkontrollierten Leitungen zum Testen von FPGA-Chips handelte und die Leiterplatte entsprechend kostspielig ist, sollte ein Redesign der Leiterplatte vermieden werden. Die vertauschten Leiterbahnen waren dünn und teilweise auf Innenlagen geroutet, was eine manuelle Nachbearbeitung sehr zeitaufwendig und risikobehaftet machte. Daher wurde nach alternativen Lösungen zur kostengünstigen Behebung des Fehlers gesucht. Letztendlich wurde beschlossen, eine kleine Interposer-Platine zu entwerfen, welche zwischen den betroffenen Steckern und der Leiterplatte bestückt werden sollte. Diese Platine sollte die vertauschten Leitungen kreuzen und alle anderen Leitungen durchkontaktieren. Teil dieser Aufgabenstellung war das Design der Interposer-Platine, wobei aufgrund des bereits fixierten Produktionstermins ein Zeitlimit von drei Tagen eingehalten werden musste. Das Leiterplattendesign sollte mit Pads Professional umgesetzt werden. Da die Umstellung auf dieses E-CAD-Tool erst kürzlich erfolgt war, wurde bisher erst ein Leiterplattendesign erstellt und viele der Funktionen des Werkzeugs waren daher neu. Dadurch war die Umsetzung des Designs in der begrenzten Zeit durchaus herausfordernd. 

\section{HMI EMV Störfestigkeitsmessung}
Bei einem HMI mit 15,6-Zoll-Touchscreen sollten zur Vorbereitung auf eine EMV-Zertifizierung in einem externen EMV-Labor Störeinkopplungsmessungen durchgeführt werden. Ziel dieser Messungen war es, den aktuellen Stand der EMV-Eigenschaften abzuschätzen und gegebenenfalls erforderliche Optimierungen vorzunehmen. Bei vorhergehenden Messungen wurden Ausfälle des LVDS-Controllers und der I2C-Schnittstelle der CPU festgestellt. Diese Messungen sollten rekonstruiert und durch entsprechende Optimierungen behoben werden.

Da es sich bei dem 15,6-Zoll-Gerät um ein relativ großes System handelt, wurde es intern aus mehreren einzelnen Leiterplatten aufgebaut, die über Flachbandkabel miteinander verbunden sind.

\section{Touch-Konfiguration für EMV-Robustheit}
Bei einem HMI zur Steuerung von Laborzentrifugen wurde festgestellt, dass es bei kabelgebundener Störeinkopplung zu ungewollten Auslösungen des Touchscreens kam. Daher sollten Änderungen an der Konfiguration des Touch-Controllers vorgenommen werden. Ziel dieser Optimierung war es, ungewollte Auslösungen des Touchscreens während EMV-Prüfungen nach Haushaltsnormen zu verhindern, während gleichzeitig ein reaktionsschnelles Benutzererlebnis gewährleistet bleibt.

\chapter{Technische und methodische Umsetzung}
%=============================================
% \color{blue}
% Hier wird erläutert, \emph{wie} die Aufgabe umgesetzt wurde. Es wird sowohl die technische Umsetzung samt verwendeter Werkzeuge und Bibliotheken (HW, SW, Messgeräte, ...) erläutert, wie auch die methodische Vorgangsweise (Studie, Modell, Prototyp etc).

% \vskip 8mm
% Umfang: ca. 2-3 Seiten 
% \color{black}

\section{Inbetriebnahme und Verifikation HMI}
Für die Inbetriebnahme der Baugruppen mussten zunächst Bootloader, Kernel und Root-Filesystem aufgespielt werden. Anschließend wurden Konfigurationen wie Netzwerkeinstellungen, die Erstellung von Datenpartitionen und die Bootkonfiguration vorgenommen. Das Aufspielen des Bootloaders auf das i.MX8-Board erfolgt über das Tool UniversalUpdateUtil (uuu). Nach dem Aufspielen des Bootloaders kann der Kernel installiert werden. Hierzu wird auf einem Arbeitsrechner ein TFTP-Server gestartet, über den die Kernel-Datei mithilfe eines firmeneigenen Softwaretools im Bootloader geladen und installiert wird. Der gleiche Ablauf wird auch für das Root-Filesystem durchgeführt. 
Um diese Abläufe nicht manuell für alle Musterbaugruppen durchführen zu müssen, wurde ein PowerShell-Skript erstellt. Dieses Skript automatisiert die notwendigen Schritte zum Aufspielen von Bootloader, Kernel und Root-Filesystem und sendet anschließend über eine serielle Kommunikation alle erforderlichen Befehle zur Konfiguration der Baugruppen. Somit müssen lediglich die notwendigen Verbindungen wie Versorgung, Netzwerk, USB und serielle Schnittstelle hergestellt werden, um die Baugruppe automatisiert in Betrieb zu nehmen. Nach Abschluss der Leiterplatteninbetriebnahme wurden die HMI-Geräte mit Displayfront, SD-Karten, Kühlkörpern und RTC-Batterien zusammengebaut, um den vollständigen Funktionstest der Baugruppe durchführen zu können. Auch für die Funktionstests wurde ein Skript im firmeneigenen Linux-Tool ''Linrpc'' (Linux Remote Procedure Call) erstellt. Mithilfe dieses Tools können alle Funktionsblöcke der Baugruppe, wie GPIO-Pins, CAN-Kommunikation und Displayfunktionen getestet und ausgewertet werden.
Um Temperaturmessungen der Baugruppe durchführen zu können, musste ein Entwicklungsmuster umgebaut werden. Da der vorgesehene CPU-Kühlkörper der Baugruppe das gesamte Board überdeckte und somit eine Messung mit der Thermokamera nicht möglich war, wurde eine Baugruppe angepasst, um das Messen der einzelnen Bauteile auf dem Board zu ermöglichen. Die interne CPU-Temperatur der Baugruppen mit dem originalen Kühlkörper wurde mit jener des angepassten Aufbaus verglichen, um sicherzustellen, dass die Werte des angepassten Aufbaus vergleichbar mit den Werten des Originals sind. Für die Messung wurde 100\% CPU Last erzeugt und der zuvor beschriebene Funktionstest für 30 Minuten ausgeführt. Es wurden Messwerte mit der Thermokamera und Messwere der internen CPU-Temperatur aufgenommen und dokumentiert.
Zur Vorbereitung auf die EMV-Prüfungen in einem externen EMV-Labor wurden Störeinkopplungsmessungen nach Haushaltsnormen (3V) durchgeführt. Bei diesen Messungen konnte ungewolltes Auslösen des Touchscreens festgestellt werden. Um dies zu unterbinden, wurde die Konfiguration des verbauten Sitronix Touch Controllers angepasst, um die Robustheit bei Störeinkopplung zu erhöhen. Dazu wurde die ''Sitronix Touch Development Util'' Software genutzt, da damit die Parameter wie Schwellwerte für Toucherkennung und weitere Filterparameter angepasst werden können. Durch Erhöhen des Schwellwertes konnte die Robustheit von unter 3V Störeinkopplung auf über 11V erhöht werden, ohne die Benutzerfreundlichkeit merklich zu reduzieren.
Um zu Verifizieren, dass die SDIO-Timing Spezifikation sowohl bei Schreiben auf die SD-Karte als auch bei Lesen von der SD-Karte eingehalten werden, musste zunächst ein Messaufbau ermittelt werden, welcher es mit möglichst geringem Einfluss ermöglicht die Signale, welche mit Geschwindigkeiten von bis zu 208MHz (SDR104) übertragen, zu Messen. Dazu wurde ein LeCroy Wavesurfer Oszilloskop mit einer Samplerate von 20GS/s verwendet. Die Datenleitungen wurden mit Differenziellen active Solder-In-Probes gemessen. Bei ersten Messungen mit Solder-In-Probes auf der Clock Leitung kam es zu ausfällen des SDIO Interfaces aufgrund der zu hohen Kapazität der Probes.Daher wurde zum Aufzeichnen der Clock Leitung eine ZS1000 active Probe verwendet. Da bei ersten Messungen an Vias mittig auf den Leiterbahnen starke Reflexionen auf den Signalen sichtbar waren, wurde ein Umbau an der Baugruppe vorgenommen, welcher ermöglichte, direkt an den SD-Karten-Kontakten zu messen.
Mithilfe der LeCroy Mask Maker Software wurden Signalmasken erstellt, welche den Bereich auf dem Oszilloskop Bildschirm kennzeichnen, welcher nicht von Signalpunkten durchschritten werden darf, da ansonst Setup/Holdzeiten oder Rise- und Fall time Zeiten verletzt wurden.

\section{Mechanische Konstruktion einer Produktionsvorrichtung}
Die mechanische Konstruktion der gefrästen Produktionsvorrichtung wurde mit der CAD-Software SolidWorks umgesetzt. Zweck der Vorrichtung für den Klebevorgang ist es, die Sockel in der richtigen Ausrichtung anzuordnen und diese wählend des Durchlaufens des Reflow-Ofens zu fixieren. Zunächst wurde bei der Vorrichtung für den Klebeprozess eine 2D-Skizze des Sockels anhand des Herstellerdatenblattes angefertigt. Mithilfe der Funktion "Skizzenmuster" konnte in einer weiteren Skizze jede einzelne Position für die Sockel definiert werden und somit die zuvor gezeichnete Skizze vervielfacht werden. Dadurch muss bei einer Änderung nur die initial erstellte Skizze angepasst werden und es wird jede Instanz der Skizze automatisch angepasst. In der Vorrichtung wurde Platz für hitzebeständige Magnete vorgesehen, um die Sockel in der Vorrichtung zu halten. Die Vorrichtung für den Lötprozess wurde nach demselben Vorgehen erstellt.

\section{Design einer Interpser Platine}
Zu Beginn wurden über Octopart Symbole und Footprints der Stecker, welche auf der Interposer platziert werden sollten, geladen. Da das Routing der Leiterplatte relativ einfach sein würde, wurde auf eine zweilagige Leiterplatte gesetzt. Die drei betroffenen Bauteile wurden jeweils doppelt im Schaltplan platziert, Grund dafür ist, dass die Bauteilfootprints auf Ober- und Unterseite der Leiterplatte platziert werden, um die Unterseite mit der fehlerhaften Leiterplatte zu verlöten und auf der Oberseite die Bauteile zu bestücken. Aus den Gerberdaten der fehlerhaften Leiterplatte wurden die Positionierungen gemessen und die  Bauteilfootprints entsprechend positioniert. Durch Übereinanderlegen der Gerberdaten des Interposers mit jenen der Originalplatine konnte die Positionierung überprüft werden und die Maße für die Außenkontur des Interposers festgelegt werde, sodass dieser mit keinen anderen Bauteilen kollidiert. Nach Platzieren der Bauteile wurden durch Vias alle Leitungen zwischen Top und Bottom Seite der Leiterplatte kontaktiert, wobei die fehlerhaften Leitungen gekreuzt wurden.

\section{HMI EMV Störfestigkeitsmessung}
Um die vorhergegangenen Messungen rekonstruieren zu können, wurde der zugehörige Messbericht zur Hilfe genommen. Laut dem Messbericht wurden bei Störeinkopplung auf den Netzwerk-, Versorgungs- und Kameraleitungen Kernel Meldungen festgestellt, welche auf einen Ausfall von I2C und LVDS Interfaces hindeuteten.
Die Ausfälle konnten bei 4V-Störeinkopplung rekonstruiert werden. Zunächst bestand die Vermutung, dass der Ausfall des LVDS-Controllers in der Folge für den Ausfall des I2C-Controllers sorgte. Es wurden unterschiedliche Maßnahmen getroffen, um die Versorgung der Leiterplatte mit dem LVDS Controller zu Verbessern, Stützkondensatoren auf den Versorgungsleitungen, schirmen der Flachbandkabel mit Kupferklebeband und eine verbesserte Masseanbindung der gesamten Leiterplatte durch engere Verbindung mit Masse bezogenen Teilen wie dem Gehäuse. Nachdem keine rekonstruierbare Verbesserung durch diese Anpassungen erzielt werden konnte, wurde durch mehrere Messungen überprüft, ob die initiale Vermutung, dass der LVDS Controller Ausfall für den Ausfall des I2C-Interfaces mitverantwortlich war, korrekt war. So konnte durch genaueres überprüfen der Kernel Meldungen festgestellt werden, dass bei Testdurchläufen zuerst eine Fehlermeldung des I2C-Interfaces zu sehen war. Mit dieser neuen Erkenntnis wurde die Fehleranalyse entsprechend angepasst. Da sich das I2C-Interface über alle durch Flachbandkabel verbundenen Leiterplatten im Gerät zog, wurden zunächst alle Flachbandverbindungen zum Mainboard getrennt und die Testdurchläufe wiederholt. Nachdem bei alleinigem Betrieb des Mainboards keine Ausfälle festgestellt wurden, wurde durch Zuschalten der anderen Leiterplatten und erneutes Testen geprüft, durch welche Leiterplatte die Störungen in das I2C-Interface eingekoppelt werden.

\section{Touch-Konfiguration für EMV-Robustheit}
Wie bereits zuvor beschrieben wurde auch hier durch Anpassen der Sitronix Touchcontroller Konfiguration und durch Erhöhen des Schwellwertes ein robusteres Verhalten bei EMV-Störeinkopplung umgesetzt. Wieder wurde die bestehende Touch-Konfiguration ausgelesen und im Sitronix Tool ''Touch Development Util'' angepasst und infolgedessen wieder über das Linux ''sitronix-tool'' welches firmenintern weiterentwickelt wurde, in den Touchcontroller geladen. Um die angepasste Touchcontroller Konfiguration in den neuen Chargen vorzusehen, wurde die Touch-Konfiguration an den Hersteller der Displayfronten übermittelt.

\chapter{Ergebnisse und Zusammenfassung}
%=======================================
% \color{blue}
% Hier werden die Ergebnisse dargestellt und die Zielerreichung diskutiert. Danach folgt eine Zusammenfassung der Erkenntnisse (Resümee) und ein optionaler Ausblick.

% \vskip 8mm
% Umfang: ca. 1-2 Seiten


% %========================================================
% % diese Seite zeigt Beispiele für die Angabe von Quellen
% % in technischen Berichten. Sie ist in der Abgabeversion
% % zu entfernen!
% \newpage
% %%%-----------------------------------------------------------------------------
% \appendix                                                             % Anahng 
% %%%-----------------------------------------------------------------------------
% \color{black}
\section{Inbetriebnahme und Verifikation HMI}
Durch die Inbetriebnahmen und Funktionstests konnte festgestellt werden, dass alle getesteten Funktionsblöcke wie erwartet funktionierten und zugleich konnte das vorbereitete Funktionstest Script in weiterer Folge für EMV Abstrahlungsmessungen in externen EMV Laboren genutzt werden, da dort die Abstrahlung bei Betrieb aller Funktionen zu überprüfen ist und bei Störeinkopplung/Störeinstrahleung geprüft werden kann ob einzelne Funktionen ausfallen.
Auch die Temperaturmessungen ergaben, dass die geforderte Oberflächentemperatur von maximal 80 °C nicht überschritten wird und wurde in einem Bericht den Anforderungen aus dem Pflichtenheft gegenübergestellt, wobei eine genaue Dokumentation des Prüfverfahrens in Form eines Worklogs angehängt wurde.
Die Angepassten Touchcontroller Konfigurationen wurden auf weiteren Testgeräten aufgespielt, um bei zukünftigen EMV-Zertifizierungen in einem externen Labor bestätigen zu können, dass es zu keinem Fehlverhalten der Geräte kommt. Bei den Messungen des SDIO-Interfaces der SD-Karte konnten keine Probleme beim Schreiben auf die SD-Karte festgestellt werden, jedoch kam es zu mehreren Verletzungen der maximal erlaubten HIGH-Threshold Spannung durch Signal-Überschwinger bei steigenden Flanken, um diesen entgegenzuwirken wurden Messungen mit Serienwiederständen auf den Datenleitungen durchgeführt und für ein Redesign 39Ohm Serienwiderstände vorgesehen.

\section{Mechanische Konstruktion einer Produktionsvorrichtung}
Um zu überprüfen, ob die Positionierung der vorgesehenen Sockelplätze mit den Positionen auf der Leiterplatte übereinstimmt, wurde das 3D-Modell der Leiterplatte mit den Sockeln in SolidWorks importiert und über die erstellte Vorrichtung gelegt. Die Konstruktionsdateien wurden an den Projektleiter übergeben und von diesem freigegeben.

\section{HMI EMV Störfestigkeitsmessung}
Durch schrittweises Verbinden und wiederholtes Testen wurde festgestellt, dass die kleinste Leiterplatte im Gerät, die zur Montage eines Helligkeitssensors in der Front des Gehäuses dient, für die Ausfälle des I2C-Interfaces verantwortlich ist. Diese Leiterplatte ist über ein ungeschirmtes Flachbandkabel mit dem Mainboard verbunden und umfasst die Versorgungsleitungen und I2C-Leitungen des Sensors. Durch eine Masse gebundenen Schirm des Flachbandkabels konnte die Störfestigkeit auf über 14V erhöht werden.

\section{Touch-Konfiguration für EMV-Robustheit}
Um die angepasste Touchcontroller Konfiguration in den neuen Chargen vorzusehen, wurde die Touchkonfiguration an den Hersteller der Displayfronten übermittelt.
Im Zuge dieser Anpassungen konnten Fehler im Linux-Tool festgestellt werden.  Die Dateien, welche Firmware und Konfiguration enthalten und vom Hersteller benötigt werden, um die angepasste Konfiguration in neuen Chargen vorzusehen, wurden über das Linux-Tool fehlerhaft ausgelesen und die Konfiguration konnte vom Hersteller nicht korrekt verwendet werden. Diese Fehler wurden mit Kollegen aus der Softwareentwicklung besprochen und eine Behebung der Fehler wurde vorgesehen.


% \chapter{Beispiele für Quellenangaben in Texten}

% In technischen Berichten ist es wichtig, korrekte Quellenangaben zu machen, um die verwendete Literatur und andere Ressourcen angemessen zu zitieren. Dies dient der Glaubwürdigkeit und Nachvollziehbarkeit der Informationen im Bericht.


% \noindent Hier sind einige allgemeine Richtlinien für die Quellenangaben in technischen Berichten:\\

% \begin{enumerate}
% 	\item \textbf{Zitierweise:} Stellen Sie sicher, dass Sie einen einheitlichen Zitierstil verwenden und sich an dessen Formatierungsvorgaben halten. Jeder Zitierstil hat seine eigenen
% 				Regeln für die Formatierung von Autorennamen, Titeln, Jahreszahlen usw.
% 	\item \textbf{Vollständige Angaben:} Geben Sie alle erforderlichen Informationen für die Quellenangabe an. Dazu gehören Autor(en), Titel, Veröffentlichungsdatum, Verlags- oder Website
% 				Informationen, Seitenzahlen (falls zutreffend) und ähnliches.
% 	\item \textbf{Nummerierung:} In einigen technischen Berichten werden die Quellen im Text durch Nummern oder Klammern gekennzeichnet. Die Nummern verweisen auf das entsprechende
% 				Literaturverzeichnis am Ende des Berichts.
% 	\item \textbf{Literaturverzeichnis:} Fügen Sie am Ende des Berichts ein vollständiges Literaturverzeichnis hinzu, in dem alle zitierten Quellen aufgelistet sind.
% 	\item \textbf{Online-Quellen:} Für Online-Quellen sollten Sie die URL und das Zugriffsdatum angeben, um die Quelle eindeutig zu identifizieren.
% \end{enumerate}

% \vspace{1cm}
% \noindent Die folgenden Sätze zeigen, wie eine Quelle angegeben wird. In der Datei \textit{references.bib} werden die einzelnen Quellen
% aufgelistet. Tragen Sie hier Ihre eigenen Quellen ein und verwenden Sie den vorgegebenen Zitierstil.\\

% % \noindent So nutzen Sie Literaturverweise. \cite{HBuch16} \\
% % So verweist man auf einen Artikel. \cite{MMuster23} \\
% % Das funktioniert auch mit Online-Quellen. \cite{onlineJohn}.\\
% % Und noch ein Beispiel für einen technischen Bericht. \cite{techSusi}\\
% % Auch ein Datenblatt sollten Sie referenzieren. \cite{datenblatt}

% \addcontentsline{toc}{chapter}{Beispiele für Quellenangaben in Texten} % Fügen Sie den Anhang zum Inhaltsverzeichnis hinzu



%=========================================
% Quellenverzeichnis erzeugen mit Eintrag im Inhaltsverzeichnis
\printbibliography[heading=bibintoc]

\end{document}
